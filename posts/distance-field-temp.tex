\chapter{Distance Field}




\section{Destruction Masking}

How do we use signed volume distance fields as one step in creating the destruction mask.

\begin{enumerate}
	\item We start by marking out the areas where we want the mask to appear on the target geometry. We do this by placing spheres(the destruction parts) on the target locations.
	\item Next we take this group of spheres and compute a signed volume distance field. We store the distance field in a volume texture so that each texel contains the shortest distance to any sphere(the destruction holes, so the surface of the distance function is the destruction entity).
	\item Using spheres makes computing the distance field a fast operation.
	\item We can also get away with a rather low resolution distance field of approximately 2 meters per pixel.
	\item The volume texture is local to the target geometry and placed to tightly encapsulate it. So we can have one volume textures for each destruction masked geometry in the scene.(the volume should only encapsulate the destruction parts, not the whole geometry, but i'm sure there should be some parameter like damege range)
	\item The spheres are authored by the artist by first auto generating a set of spheres so that each destructable part has one(yes, it was what I thougnt). The artist can then go on and move/scale the spheres as well as add new ones. For example, to cover a hole that is not close to square the artist might want to put several smaller spheres to get a better fit(with no cost, because the volume will be precompute).  
\end{enumerate}


The next step is to use the low resolution distance field to get a detailed destruction mask:
\begin{enumerate}
	\item Here we can see a destroyed part of a building wall and the distance field around it drawn with point sampling(Because the resolution of the volume texture is very low).
	\item Turning on trilinear filtering gives a smooth gradient, the surface of the destruction mask is at gradient value 0.5.
	\item By multiplying and adding an offset we can find the surface of the destruction mask(from the volume texture). Additionally we project a detail texture on to the geometry and use it to offset the distance field. This breaks up the low frequency shape of the mask and creates a more interesting look.(So destruction mask is an another geometry, maybe the child of the main geometry, the meshes of the destroyed part has been removed)
	\item Final result with texturing. 
\end{enumerate}


\subsection{Volume Texture Considerations}
One thing to consider when working with small textures on XBox 360 is that texture dimensions need to be a multiple of the tile size. For 3D textures the tile size is 32x32x4.

Our distance field textures are typically 8x8x8 in dimension which will be a considerable waste of memory if we allocate them one by one and have a lot of them.

Our solution to this is to use texture atlases.

To simplify packing textures into the atlas we keep one atlas per texture dimension.

As always when working with texture atlases we need to add padding between the textures in the atlas to prevent leakage across texture borders. Since we don't need to generate mipmaps for these textures(they are so low resolution that LOD them before min filtering would be needed) padding only needs to be one pixel wide.

\subsection{Drawing the Destruction Mask}
When we draw geometry with destruction masking we first sample the distance field, if we then see that the pixel is outside the mask volume we can use dynamic branching to early out from the rest of the mask computations. This saves us a lot of computation time for pixels not affected by the mask.

However, when we tried this on PS3 we saw that dynamic branching was not very efficient. The RSX has a fixed cost of 6 cycles when introducing a branch in the shader, and for the branch to be taken a group of up to 1600 pixels all need to take the same path.

This led us to try out a different approach, drawing the mask as a deferred decal.

\subsection{Deferred Decals}
Deferred decals works similar to how we apply light sources in a deferred renderer.

\begin{enumerate}
	\item Begin by drawing the main geometry of the scene to fill in the G-buffer(as before).
	\item Next draw a convex volume covering the decal area(around the decal area).
	\item When drawing the volume fetch depth from the depth buffer and convert to local position within volume.
	\item Knowing the position within the volume we can look up the decal opacity by for example sample it from a volume texture.
	\item Finally we alpha blend the result to the G-buffer.
\end{enumerate}  

One advantage of this method is that the convex volume we draw covers only the decal area, and pixels outside of it are unaffected. So if we can make a good fit for the convex volume for the decal area we can save performance. Many optimization techniques has been developed for applying light sources in a deferred renderer(stenciling, tile classification) and we could potentially use those here as well.

\subsection{Projecting Detail Texture}
The destruction mask replaces the underlying diffuse albedo and normals by projecting detail textures onto the masked geometry.

In order to achieve that we need tangent vectors for the projected surface -  both for converting world space positions into texture coordinates, and to convert tangent space normals from the detail normal maps to world space normals.  

It is appealing to construct tangent vectors from G-buffer normals, but those normals are typically based on normal maps, while we want then normals from the actually geometry.

We took a look at the geometry our artists are creating and saw that there are only a few tangent vectors required for a typical object. With that in mind, we created a look up table with the required tangent vectors and stored them in a texture. Then, when drawing the scene geometry we write an index to the G-buffer. We sample this index when drawing the deferred volume and use it as offset into the lookup table.

\subsection{Alpha Blending and g-buffer}
\subsection{Distance Field Triangle Culling}
Another PS3 specific method we have explored to optimize drawing the destruction mask is distance field triangle culling.

Instead of branching in the fragment shader, which we saw is not very efficient on the PS3, we have moved the branch from the GPU to the SPUs and do the branch per triangle.

The way it works is that each triangle in the mesh we test it against the distance field to see if it inside or outside the mask volume. We output the triangles into two separate index buffers and issue two draw calls. The draw call for triangles inside the distance field volume is drawn with a shader that computes the mask while outside triangles are drawn as normal.

To help us do this in an efficient way on the SPUs, we take advantage of the PlayStation edge geometry library and run it a step of that geometry pipeline.

We only need to do the actual culling against the distance field when the distance field changes, in between we can cache the result. 
















\section*{What is Distance field}
in an approximate sense, it will serve as our means of rapidly computing the visibility of the sky from any point in the scene we wish to light.

\section*{How to Generate Distance field}
\section*{usage}
\subsection*{intersection}
\subsection*{Ray Tracing}
\subsection{Ray Marching}

Sphere tracing 
Ray marching

\section{Surface representations}

A key observation is that in many cases, a scene will be made up of a number of (possibly moving) objects. Each object need only be able to rapidly compute its own local SDF, before being composited into the final volume texture using the ‘min’ blending mode of the GPU to generate the ‘global’ SDF.

Depending on the type of object O, we compute its SDF in one of several ways:
1. For a cuboid or ellipsoid, the SDF can be computed analytically.
2. For a rigidly deforming body with a complex shape, the SDF can be precomputed and stored as a volume texture in the local space of O
3. For a ‘height field’ object whose surface can be described as a single valued function of 2 coordinates (x, y), the SDF can be approximated directly from a blurred copy of the height field data describing the surface shape.
4. For a star shaped object (such as that shown in Figure 9) where the surface can be described as a single valued function giving the radius of the object along the direction to the object’s local origin, the SDF can be approximated directly from a blurred copy of the ‘height field’ / Z buffer stored in a cubemap.

Signed distance fields can be computed analytically for simple shapes – and we shall be returning to this shortly – or tabulated in grids, octrees, or other spatial data structures. In this course we’ll be using the simplest mapping to a GPU – namely an even sampling of the SDF over a volume, storing the values of the function in a volume texture. This kind of direct GPU representation is also used for a particular type of efficient GPU ray tracing called ‘sphere tracing’ – as used in the implementation of per-pixel displacement mapping in [Donnely05].

Analytic computation
precomputed (3D Texture, Octree/KdTree)(static scene)
The SDF of the mesh was pre-computed on the CPU once and uploaded as a 64x64x64 volume texture to the GPU. 

procedurally


We will deal with efficient generation of the SDF on the GPU later

\subsection*{differentiation from implicit equation means distance field?}

there exist two main representations of surfaces that are used in geometric modeling and computer graphics: parametric and implicit.

These, as men- tioned by Menon have advantages over parametric surfaces. One of them being, that the surface to surface intersections calculations, as the author states: “are much more computationally (numerically as well as topologically) tractable”. Menon mentions that an implicit surface can be defined as the zero contour of a function, that is:

When considering distances to objects or surfaces usually one deals with the notion of a distance field. As indicated in [JBS06] a distance field is a repre- sentation of space where each point in that space can be described by a value of a distance from that point to the closest point on a surface belonging to that space. The authors discuss methods for generating distance fields, their applications and representations. One form of representation of a distance field is storing the distance values in a voxel grid. The distance based function de- scribed by (2.12) is another form of representation of a distance field which in this particular case contains only the distance information to the surface of a sphere.

\subsection*{generate distance field}
\subsection*{union two distance field(two object bounds)}
• Combination of (instanced) distance fields can be done by taking the min of the distance fields involved.
• Instance transformation can be done by inverse transforming the domain (the input to the distance function).


Eliminating empty space certainly aids rendering efficiency, but the major benefit of partitioning is that it allows the imposition of local bounds on the Lipschitz constants yielding tighter signed distance bounds. Octree partitioning has been used in the polygonization [Bloomenthal, 1989] and ray tracing [Kalra \& Barr, 1989] of implicit surfaces. Sphere tracing reaps the same benefits from spatial partitioning as did the root finding method in [Kalra \& Barr, 1989], which used the Lipschitz constant to cull octree nodes guaranteed not to intersect the implicit surface.

Ray intersection with an implicit surface defined by a signed distance bound is penalized by the section of the domain where the gradient magnitude is greatest. Chopping an object into the union of smaller chunks allows each chunk to be treated individually, penalized only by the largest gradient within its bounds. Since the partitioning algorithm in [Kalra \& Barr, 1989] required only a bound on the Lipschitz constant of the function, the use of this octree in no way restricts the domain of functions available for sphere tracing.

Octree partitioning further enhances sphere tracing of unions and lists by optionally storing an index to the object closest to the cell. An object is closest to an octree cell if and only if it is the closest object to every point in the cell. Under this definition, some cells may not have a closest object. By the triangle inequality (19), an object is closest to a cell if the distance from the cell’s centroid to the object, plus the distance from the centroid to the cell corner, is still less than the distance from the centroid to any other object.



Domain repetition
Domain distortion
Blending fields
Adding details

\subsection*{distance field functions}

\section{Rendering Solutions}

there exist several methods for rendering implicit surfaces. One of them is the polygonization of implicit surfaces which allows to take advantage of the graphical pipeline and rasterization techniques. However, this process of representing the surface by the use of polygons might result in the loss of detail of the surface and may not accurately detect disconnected sections of the surface.

As another method Hart mentions that ray tracing is a noteworthy technique for visualizing implicit surfaces. In ray tracing the ray equation is given by the form:

The root finding methods have a drawback of finding usually more than one intersection point of the ray and the implicit surface. From all the traversal t values that satisfy the equation (2.17) only the smallest one is taken into consideration. Ray marching and sphere tracing mitigate this by only focusing on determining the first root of the aforementioned equation. The process of doing so is elaborated in the following sections.

\subsection{Ray Marching}
Note the very expensive marching on the object edges.
\subsection{Sphere tracing}


\section{Shadow}
\subsection{Hard Shadow}
\subsection*{Soft Shadow}
• Fake and fast soft shadows.
• Only 6 distance evaluations used instead of casting hundrends of rays. • Pure geometry-based, not bluring.
• Recipe: take n points on the line from the surface to the light and evaluate the distance to the closest geometry. Find a magic formula to blend the n distances to obtain a shadow factor.


Unreal Engine: Ray Traced Distance Field Soft Shadow

\section{Ambient Occlusion}

We wish to incorporate sky shadowing effects on any point P, from all objects in the scene, whatever their distance from the point P, not just the effect of nearby concavities.Objects at a greater distance should have a lesser, ‘blurrier’ effect than those near the point P.

Ambient Occlusion only be used to calculate the occlusion of distant light like sky light? This means the environment map of the house cannot use AO? 

• In a regular raytracer, primary rays/AO cost is 1:2000. Here, it’s 3:1 (that’s almost four orders of magnitude speedup!).
• It’s NOT the screen space trick (SSAO), but 3D.
• The basic technique was invented by Alex Evans, aka Statix (“Fast Approximation for Global Illumnation on Dynamic Scenes”, 2006). Greets to him!
• The idea: let p be the point to shade. Sample the distance field at a few (5) points around p and compare the result to the actual distance to p. That gives surface proximity information that can easily be interpreted as an (ambient) occlusion factor.

\section*{3D textures as implicit surfaces}
\section*{Height Map}

\section{Deform dynamically}
precomputer and compute runtime
regenerating distance field in runtime with object deformed.

\section*{Antialiasing}

Point sampling the SDF and then reconstructing it trilinearly introduces high frequency aliasing, visible in the output (and Figure 5) as slight banding across the mesh surface. These artifacts can be greatly reduced by pre-filtering the SDF by low-pass filtering it with (for example) a separable Gaussian low-pass filter. An additional advantage of this pre-filtering step is that the width of the Gaussian filter allows us to effectively choose the characteristic scale of the concavities that the algorithm highlights. We will use this feature in the next section.

Methods for computing distance fields on graphics hardware fall into two different approaches: distance meshing or scan conversion of bounded volumes.